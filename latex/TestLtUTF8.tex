\documentclass[a4paper]{article}

\usepackage[utf8]{inputenc}
\usepackage[L7x]{fontenc}
\usepackage[lithuanian]{babel}
\usepackage{lmodern}

\title{Europos Sąjungos valstybių skirtumai vertinant šalies politinę ir ekonominę situaciją}
\date{Pradinė duomenų analizė}
\author{Justyna Ziminskaja, Alina Lisovec \\ Darbo vadovas: prof. V. Čekanacičius}
\begin{document}
\maketitle


\begin{enumerate}

\item  Duomenų suradimas ir apdorojimas

\begin{itemize}
\item Duomenis gavome iš http://ess.nsd.uib.no/ess/round4/
\item Suradome Lietuvos, Estijos ir Švedijos duomenis, ir atsisiuntėme SPSS formatu. Atrinkome tokius kintamuosius: agea(respondentų amžius), gender(lytis, kur 1-moteris, o 2-vyras), happy(laimingumas), stfgov(kiek respondentas yra patenkintas savo šalies vyriausybe), stfdem(kiek respondentas yra patenkintas demokratijos būsena šalyje), trstpr(pasitikėjimas šalies parlamentu), trstlgl(pasitikėjimas teisine sistema), trstplt(pasitikėjimas politikais), trst(pasitikėjimas partijomis), trstun(pasitikėjimas Jungtinėmis Tautomis), trstep(pasitikėjimas Europos Parlamentu).
\newline Kintamieji stfgov ir stfdem matuojami 10 balų skale nuo 0 (itin nepatenkintas) iki 10 (itin patenkintas). Kintamieji trstprl, trstlgl, trstplt, trstprt, trstun matuojami 10 balų skale nuo 0 (itin nepasitikiu) iki 9 (itin pasitikiu). Kintamasis happy matuojamas 10 balų skale nuo 0 (itin nelaimingas) iki 9 (itin laimingas).

\item Mus domina respondentai kurie yra 35 ir mažiau metų. Todėl atitinkamai atrinkam tokius respondentus.
\item Nukopijuojame duomenys į txt failą.

\item Mūsų pagrindinis kintamasis t.y. satisfaction yra vidurkis dviejų kintamųjų stfgov ir stfdem

\end{itemize}

\item Lietuvos duomenų analizė

\begin{itemize}
\item Nuskaitome duomenys su R programa: 
\newline
\textit {lt=read.table('C:/Documents and Settings/Justyna Z/Desktop/kursinis/lt.txt')}
\ Išskiriame kiekvieną kintamąjį:
\newline
\textit {>gndrlt=lt[,2] 
\newline >lhappy=lt[,3] 
\newline >lsftgov=lt[,4]
\newline >lstfdem=lt[,5]
\newline >lprl=lt[,6]
\newline >llgl=lt[,7]
\newline >lplt=lt[,8]
\newline >lprt=lt[,9]
\newline >lun=lt[,10]
\newline >lep=lt[,11]}

\item Sukuriame pagrindinį kintamajį satisfaction:
\newline 
\textit{sat=(lsftgov+lstfdem)/2}
\item Sukuriame matricą visų pasitikėjimo kintamųjų:
\newline \textit {lall=cbind(lprl,llgl,lplt,lprt,lun,lep)}
\item Apskaičiuojame koreliaciją tarp mūsų pagrindinio kintamojo \textit{sat} ir kitų kintamųjų:
\newline  \textit{cor(sat,lall)}  -  atitinkamai sat su lprl koreliacijos koeficientas lygus 0,6258285,  su  llgl -  0,531893, su lplt - 0,5809648, su lprt - 0,5627991, su lun - 0,1621104, su lep - 0,2464673
\newline Koreliacija satisfaction ir  lhappy yra 0,279422, tai padrėme su komanda 
\newline \textit{cor(sat, lhappy)}
\item Matome, kad satisfaction gana reikšmingai koreliuoja su visais kintamaisiais.

\item Apskaičiuuojame vidurkius visų kintamųjų
\newline \textit{ >mean(sat) 
 \newline [1] 2.969589  }
\newline \textit{ >mean(lhappy) 
 \newline [1] 6.855098  }
\newline \textit{ >mean(lprl) 
 \newline [1] 2.31127  }
\newline \textit{ >mean(llgl) 
 \newline [1] 3.420394  }
\newline \textit{ >mean(lplt) 
 \newline [1] 2.064401  }
\newline \textit{ >mean(lprt) 
 \newline [1] 2.014311  }
\newline \textit{ >mean(lun) 
 \newline [1] 5.389982  }
\newline \textit{ >mean(lep) 
 \newline [1] 5.030411  }

% cia nzn ka pakomentuoti del vidurkiu :D 



\end{itemize}

\item Estijos Duomenų analizė
\item Švedijos duomenų analizė
\end{enumerate}

\showhyphens{Nebeprisikiškiakopūsteliaudavome}


\end{document}

\documentclass[12pt, a4paper, lithuanian]{article}
\def\LTfontencoding{T1}

\usepackage[OT2,\LTfontencoding]{fontenc}

\usepackage[lithuanian]{babel}

\usepackage{ucs}

\usepackage[utf8x]{inputenc}

\usepackage{epsfig}



\title{Europos Sąjungos valstybių skirtumai vertinant šalies politinę ir ekonominę situaciją  }


\author{Ekonometrija 3kursas\\  Justyna Ziminskaja, Alina Lisovec}


\maketitle





 

\begin{document}


% Būtina išspausdinti bent vieną unicode raidę, kad tituliniame puslapyje geria išspausdintų unicode raides
\PrerenderUnicode{ė}











\flushleft
\begin{enumerate}
\item  Tikslas sužinoti, kas įtakoja žmonių pasitenkinimą šalies politiniu klimatu.
\item Duomenis gauname iš ess/nsd.uib.no/ess/round4/

\begin{itemize}
\item  Mus domina Lietuvos, Estijos ir Švedijos duomenys
\item Pagrindiniu kintamuoju bus pasitenkinimas šalies politiniu klimatu (satisfaction,sat)
\item Sudarysime sat  regresiją atžvilgiu kitų kintamųjų, tokių kaip: laimingumas(happy), pasitikėjimas politinėmis institucijomis(trust all), pasitikėjimas teisine sistema, demokratijos būsena ir taip toliau... 
\end{itemize}
\item Sudarome kiekvienos šalies regresinį modelį
 \begin{itemize}
 \item Parenkame tinkamiausią modelį pagal standartinius reikšmingumo lygmenis(tokius kaipR-kvadratas, chi kvadratas,ANOVA,..)   
\item Paimti vienos šalies(1) modelį, pritaikyti kitos šalies(2) duomenim ir patikrinti ar kitos šalies(2) duomenis gali būti aprašyti svetimos šalies(1) modeliu. 
 \end{itemize} 

\item Gavus regresijas mes matysime nuo ko priklauso mus dominantis kintamasis(satisfaction)( su kuo koreliuoja), galbūt nuo lyties, amžiaus, partijos, religijos.
\item Lyginsime šalis, t.y.  kuo esame panašūs, kokie skirtumai, gali būti, kad pasitenkinimas priklauso nuo visiškai kitų dalykų(su Norvegija pvz.)
\end{enumerate}

\bibliography{main}








\end{document}